\section{Conclusions}
In this report no new information is presented.
The report is an overview on how centralised and decentralised systems cope with different threats.
The threats are composed in three different threat models defining three different scenarios.
It's clear that defining just one threat model and making sure that all corner cases are covered is a demanding task and a time consuming process in itself.
When writing this report it was early on decided that the focus should be put on the comparing centralised and decentralised system instead of trying to define the perfect threat models.
For future research defining the different threat models in more detail is a first step.

The comparisons made show that centralised and decentralised systems have their pros and cons.
No one can say that centralised or decentralised systems will be better nor worse in dealing with all types of threats.
In the threat models defined in this report decentralised systems have a small advantage.
As an example to this in the aftermath of a natural disaster a community can on their own set up a decentralised network without having to wait for someone to fix a centralised node e.g. setting up a temporary cellular tower.
This would mean that a more organised and controlled recovery would start sooner with a decentralised network than a centralised network.

As a last note it is important to remember what is trying to be achieved.
In the threat model about privacy it is a must to define what type of privacy you discuss because it is a big difference between "institutional privacy" and "social privacy".
