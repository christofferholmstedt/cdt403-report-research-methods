\section{Natural disaster}
During the last decade the evolution of cell phones has gone from just being mobile phones to todays "smartphones".
The smart phones can do a wide variety tasks depending on which applications are installed.
The last years rise in smart phone usage has made the use of social media expand to all parts in society \cite{taylor2012}.

The penetration rate of social media has almost made the use of social media by governments and different agencies a must, for communication to its citizen, especially in crises.
As an example The Queensland Police Service in Australia used their facebook page successfully during several floodings and thunderstorms in 2011.
Their success is shown by the number of "likes" of their facebook page that grow "[...] from under 20 000 to over 160 000 within three days of flash flooding [...]" \cite{taylor2012}.
A like can mean a lot of things, everything from a person just want to appreciate what they (the recipient of the like) do to someone who really found the information posted valuable.

Other ways of using social media is by citizens that can create their own networks and communicate with each other.
During the floodings in Queensland and Victoria (Australia) in 2010 and 2011 several of these networks were created \cite{bird2012}.
The reports that came in to these networks and in some cases republished (shared by more than the original author) were often about the latest news from agencies and traditional news such as TV and radio.
Though this was not the only way of using the different networks in social media, one main usage of social media is the communication between citizen e.g. citizen that has moved from one area to another can report about road blocks as well as which road are available to travel on.
The updates between citizens often include more local information than the regular news reports \cite{bird2012}.

In \textit{"The role of social media as psychological first aid as a support to community resilience building"} \cite{taylor2012} the conclusion is drawn that social media can function as "psychological first aid (PFA)" \cite{taylor2012}. 
\begin{quote}
\textit{"The role of social media in
this context is not to replace face-to-face support or contact, or to replace official warning services, but it can expand capacity to deliver information, extend the reach of official messages and limit the psychological damage caused by rumours and sensationalised media reporting."} \cite{taylor2012}
\end{quote}
This shows that social media such as facebook has gone from being just a community where you exchange basic status updates and comments to be have a role in emergencies.

Furthermore Lawson et al \cite{lawson2012} writes about cell phones as a part of disaster preparedness.
The qualitative research conducted took place in Haiti where several natural disasters occur every year.
During disasters landlines most often goes down which could happened to cellular towers as well but is not as usual according to one participant in the survey.
A more likely scenario is that the cell networks get overcrowded after a disaster when everyone with a cell phone tries to contact relatives and friends.
The overcrowded traffic makes it near impossible to get through with regular phone calls but data traffic does.

With this background it's clear that in the world today technology such as cell phones and services such as Facebook, Twitter and Google plus have a big role when it comes to getting a community back on track after a natural disaster.
It's now time to recap the threat model mentioned in earlier chapter.
The following is the scenario given:
\begin{itemize}
  \item The only devices left operating are the ones with battery such as laptops and cell phones.
  \item These devices have one or more features that allows them to connect to other similar devices, example of this could be WiFi, Bluetooth, IR or USB drive.
\end{itemize}
The question now arise how would a centralised system coup with this as well as how would a decentralised system do?

Centralised systems such as cell phone networks as well centralised services such as Facebook depend on backup systems.
Cell towers requires emergency power from a reserve power unit or a generator of some kind.
The cell towers also needs a connection out to the cell network for communication to be possible.
In a centralised system like cell networks it's clear that these towers are a weak point in the chain, exactly how weak is hard to say.

On standby network operators have "cellular on wheels" or COWs in short.
COWs are basically a mobile cellular tower.
These COWs are moved in preparation of a thunderstorm to the locations that will most likely be struck hardest.
COWs help the network to get up and running as soon as possible, even if it's not at its full capacity it is at least some traffic that goes through \cite{swartz1999}.

The next step in the chain is the services which are most likely hosted somewhere far away from the disaster area and still up and running.
Even if the disaster would strike where the services are hosted their will probably be redundancy geographically in another part of the world which would still be operational.

Research has in the last years been conducted with the goal to use peer-to-peer network for services such as sharing content, video and music \cite{web:tribler}, as well as creating a social network \cite{web:peerson}.
It's a logical step to move towards a peer-to-peer social media network which could be used in the case of emergency.
If a peer-to-peer or ad-hoc network were possible to set up straight after a natural disaster between the survivors the initial communication could commence straight away.

Another possibility with a peer-to-peer network is that it could be designed in a delay tolerant fashion.
This would mean that information could spread to others over the course of hours or days even if the sender and receiver doesn't have access to the network at the same point in time.
As an example to demonstrate this imagine a disaster strikes a village.
They use an already existing peer-to-peer network or set up a new one to organise the recovery work in the village.
After a few days the first rescue workers reach the village and to get an up to date picture of the situation they connect to the local peer-to-peer network and get all information needed in a few seconds. 
The network in itself would in this case act as a logging system of all the work that has been done and what needs to be done in a decentralised fashion.


% Section about Natural disaster threat model \cite{taylor2012,lawson2012,bird2012}
