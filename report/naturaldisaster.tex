\section{Natural disaster}
During the last decade the evolution of cell phones has gone from just being mobile phones to todays "smartphones".
The smart phones can do a wide variety tasks depending on which applications are installed.
The last years rise in smart phone usage has made the use of social media expand to all parts in society \cite{taylor2012}.

The penetration rate of social media has almost made the use of social media by governments and different agencies a must to communicate to its citizen, especially in crises.
As an example The Queensland Police Service in Australia used their facebook page successfully during several floodings and thunderstorms in 2011.
Their success is shown by the number of "likes" of their facebook page that grow "[...] from under 20 000 to over 160 000 within three days of flash flooding [...]" \cite{taylor2012}.
A like can mean alot of things, everything from a person just want to appreciate what they (the recipient of the like) do to someone that really found the information posted valuable.

Other ways of using social media is by citizens that can create their own networks and communicate with each other.
During the floodings in Queensland and Victoria (Australia) in 2010 and 2011 several of these networks were created \cite{bird2012}.
The reports that came in to these networks and in some cases republished (shared by more than the original author) were often about the latest news from agencies and traditional news such as TV and radio.
Though this was not the only way of using the different networks in social media, one main usage of social media is the communication between citizen e.g. citizen that has moved from one area to another can report about road blocks as well as which road are available to travel on.
The updates between citizens often include more local information than the regular news reports \cite{bird2012}.

In a survey conducted it shows that social media can function as "psychological first aid (PFA)" \cite{taylor2012}. 
INSERT BLOCK QUOTE FROM TAYLOR FINDINGS ...starting from "The role of social media in this context"...finish. Continue then with the haiti report.

% Section about Natural disaster threat model \cite{taylor2012,lawson2012,bird2012}
