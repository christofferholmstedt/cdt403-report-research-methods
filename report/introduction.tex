\section{Introduction}
Defining good threat models can be very helpful in improving existing systems, or contribute to building better, safer and more fault tolerant systems. Threat models are especially important in Software engineering and Computer Science in general because it is a relatively new field of research and there are not much threat models defined. A threat model can help us to understand how a particular system reacts to a situation that is different from normal operating conditions. The limits that a threat model define makes it easier to discuss and reach conclusions in a defined scenario.

In our paper, the threat models are picked up in that way so we can use them to compare the ability of the centralised and decentralised computer systems to deal with circumstances other than normal. Because centralised and decentralised systems have much broader meaning, we have narrowed our focus to centralised computer networks and decentralised or distributed computer networks and applications that are dependent on these particular network architectures. [I need some help here to find suitable references to the decentralized and centralized systems] To study the characteristics of the centralised computer systems, we will use well know centralised web applications. 

The most popular now are the social networking sites, where the most popular of them Facebook has recently passed one billion users \cite{web:facebookpassesbillion}. Other popular social networks are Twitter, Google Plus and many others. Conclusions for decentralised computer systems will be made after observing how our defined threat models affect Peer-to-peer networks. The field of decentralised computer system is much broader field than just peer-to-peer networks, but for purpose of our paper we assume that it is enough. There is also a set of other application in between centralised and decentralised model like e-mail, DNS and XMPP that have both centralised and decentralised characteristics. They are not subject of interest in this paper.

In modern society the computer networks have become the main medium for sharing information. The  biggest network known to mankind is the Internet. It is practically like a super organism build on top of a smaller building blocks – the computer networks themselves. Since the invention of the World Wide Web in 1990s by Sir Tim Berners-Lee \cite{web:timbernerslee} the Internet has started its dominance over other mediums for information sharing like the Television or the Radio. According to Internet World Statistics research \cite{web: internetworldstats}, statistics show that there were approximately 361 million Internet users on 31 December 2000 and on the same date 2011 there were 2,27 billion active users on the Internet, with a total world population of around 7 billion. This clearly shows the potential of this medium for spreading information, because roughly every third person on the planet can access the information. The WWW has transformed the entire Internet from a simple tool for communication to a revolutionary technology which penetrates to the core of every society. According to the same statistics \cite{web:internetworldstats} the penetration percent in the most progressive countries can go even above 60 \%. Examples are Northern America (78.6 \%), Australia and Oceania (67.5 \%) and Europe (61.3 \%). Roughly every third person on the planet has access to the Internet, and has power to share information. 

Statistics for internet penetration in human societies, implies that people need this tools even in the most harsh conditions. Especially important are problems which can occur in the infrastructure of the computer systems or threats related to data integrity. We have modelled our threat models, to target both of the mentioned aspects of potential faults in computer networks. The most important thing is that we do not try cover the whole picture of a threat, but rather than that we take certain characteristics and trough them we try to give a model of a situation.

We have structured the paper in manner that we think is the most suitable for reading it without going back and forward?. Fist we explain our methods for gathering our information. We present which databases for paper search are used and the phrases we used to generate search results. Beside this, we explain how we choose out threat models, and how we have modified the according to our needs. Than we give a detail description for every situation that is related to a threat model, and try to study the systems behaviour. On basys of this studies we conclude how good or bad is particular system with dealing that potential threat.
 
First with our natural disaster threat model, we try to replicate some of the conditions that can occur during the most common natural disaster events -  foods. In this conditions manly the infrastructure is disabled, due to electrical problems of some cable failure. By applying this threat model to both centralised and decentralised networks we can see make assumptions how the systems would react on them. After that we can summarize the results and give a general conclusions which system is better in dealing with this potential threats.

To see how data present in the computer networks can be compromised we have chosen models that target that specific characteristics. By applying privacy threat model we can assume how vulnerable is  users of centralised and decentralised networks, and in which potential ways it can be exploited. We try to find if there is any specific characteristics in any of the systems that can successfully deal with this threat.

Our last threat model addresses censorship in computer networks. Consequently the rise of the popularity of some of the applications, there is exponential growth of the data present in the computer networks that is shared between the users. As we all know, users tend not to care too much of what type of data they release into the network, and there must be some mechanisms to protect innocent people from that type of information. But, it has become a common practice for authorities and the governments to use this misbehaviour of the users as an excuse to 	impose censorship over the users data and to limit their freedom of speech. We will apply our threat model to the computer networks and try to find is there any practical solution to this problem. 

We conclude our paper by summarizing the results and by giving complete a brief overview of our work and what we have tried to achieve.
