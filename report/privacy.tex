\section{Privacy in Social Network Services}
As stated in earlier chapters social media is on the rise.
Most of these networks are centralised e.g. Facebook, Twitter and Google plus.
Users share any kind of private information in the belief that it is their own information only accessible by friends that they know.
This is though far from true in many cases.
As an example social media websites search through all information submitted by users to be able to target the user with tailored ads.
This is not appreciated by most americans as a telephone survey conducted in 2009 conclude \cite{turow2009}.
It's not far fetched in this scenario to believe that similar results would be acquired in other parts of the world.

Other studies show that "institutional" privacy e.g. the scenario above when a company analysis its users behaviour and shared information is not the main concern \cite{raynes-goldie2010}.
Instead the users in the study thought that "social" privacy is more important.
Social privacy include scenarios from friend requests (from the "wrong" person e.g. the users boss or teacher) to posting daily messages that should be suitable for all friends.
Some participants in the study was troubled by the lack of control of shared data on facebook.
As an example it's not often that the same message is perceived in the same way by your boss and your best friend.
To take the example one step further a message posted during a night out with friends in the weekend could be misinterpreted by the users boss and have consequences the next workday.

It's important to remember that what we perceive as problems with privacy in different services one day may not be a problem the next day.
In 2006 Facebook introduced the news feed function or the Facebook wall as it's called today.
With this release all information available in every users network of friends got aggregated to one simple list of actions taken during the last period of time.
The information was available all the time but not that easy to get an overview of, now the news feed became that overview and it became the main feature of Facebook.
Today, in 2012, facebook groups form for all kinds of reason and this was also the case back in 2006.
In a few days the group "Students Against Facebook News Feed" had over 700 000 people \cite{boyd2008}.
This clearly shows that privacy is a main concern for many people though privacy can come in different shapes and forms.
The news feed is still the main feature on Facebook today after some privacy settings were added.

One interesting feature about privacy is that users that know about loopholes in security will use them even if the same action against themselves is not appreciated.
Raynes-Goldie found that participants in her study used direct URLs to images on Facebook that was normally hidden behind privacy settings \cite{raynes-goldie2010}.

With the centralised systems covered it's time to look at the decentralised systems.
The basics of a decentralised system are the lack of central nodes, a central authority that controls the network and the ability to scale very well.
In \textit{"Friendly Surveillance - A New Adversary Model for Privacy in Decentralized Online Social Networks"} \cite{greschbach2012} the authors defines services such as Facebook and Google plus to be a "Social Network Service (SNS)" and decentralised versions are named "Decentralised Online Social Network (DSON)".
It is important to remember that traditional SNS have been online for several years but it's only recent years that researcher has started to look at DSONs.
The move from a SNS to a DSON effectively remove the earlier mentioned institutional privacy problems but on the downside also introduce several new problems.

The first problem is the question on how to store data.
With the lack of a central node to store the information it must be stored somewhere else.
A first step would be to encrypt all data and store it on all your friends computers so at least when you go offline others can access it through your friends that are online.
The problem here now arise that a friend of you could monitor the traffic that comes in and request particular pieces of information.
Even if the information is encrypted metadata is available such as how much disk space the information takes and how often a specific node requests similar data.
With the information about disk space usage your friend could find out if it's text, music or videos that are shared.
With the information about how frequently a node requests some information and during which hours of the day the sender could compose a pretty clear picture on the behaviour of that node \cite{greschbach2012}.
It's also in this area Greschbach et al \cite{greschbach2012} comes to the conclusion that a good technical solution that doesn't give away too much information in metadata is needed.

The second question is about trust.
How do a user trust another user.
In a centralised system users trust the centralised service as an authority or "third-party" so if user A sees user B online user A can in some sense be sure that it's user B that is online.
This is because users indirectly approve of the authentication system used by the service.
In decentralised systems there is no authority that vouch for someone else instead decentralised system can use some kind of reputation.
User B could choose to improve the reputation of User A.
If then User C comes in contact with User A the reputation set by User B (together with many others) gives a representation on how well User A behaves in the network and how much trust can be given to User A.
The problem with a system like this is that reputation or trust is often depending on context \cite{mondal2006}.
User B might trust User A when it comes to talking about some soccer games that took place yesterday but User B might not trust User A when it comes to talking about private information such as medical status.
Another aspect is that the context also differs between users.
A generalization in a reputation system as explained above clearly has it drawbacks and it's in this area , "effective trust models" Mondal et al \cite{mondal2006} suggests that the research community can contribute.

Tightly coupled with trust is also accountability.
If one user in the network has bad behaviour it will probably be easy to deny that user further access to whatever resource is requested.
Though one of the main features of many decentralised systems is the possibility for a user to be completely anonymous.
The possibility to be completely anonymous makes it possible for the user that has bad behaviour in the network to act as a new user when the first one has been blocked by a noticeable amount of users.
In the end to introduce accountability to a decentralised system might not what you want to do due to low amount of malicious users \cite{mondal2006}.

