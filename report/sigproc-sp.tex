% THIS IS SIGPROC-SP.TEX - VERSION 3.1
% WORKS WITH V3.2SP OF ACM_PROC_ARTICLE-SP.CLS
% APRIL 2009
%
% It is an example file showing how to use the 'acm_proc_article-sp.cls' V3.2SP
% LaTeX2e document class file for Conference Proceedings submissions.
% ----------------------------------------------------------------------------------------------------------------
% This .tex file (and associated .cls V3.2SP) *DOES NOT* produce:
%       1) The Permission Statement
%       2) The Conference (location) Info information
%       3) The Copyright Line with ACM data
%       4) Page numbering
% ---------------------------------------------------------------------------------------------------------------
% It is an example which *does* use the .bib file (from which the .bbl file
% is produced).
% REMEMBER HOWEVER: After having produced the .bbl file,
% and prior to final submission,
% you need to 'insert'  your .bbl file into your source .tex file so as to provide
% ONE 'self-contained' source file.
%
% Questions regarding SIGS should be sent to
% Adrienne Griscti ---> griscti@acm.org
%
% Questions/suggestions regarding the guidelines, .tex and .cls files, etc. to
% Gerald Murray ---> murray@hq.acm.org
%
% For tracking purposes - this is V3.1SP - APRIL 2009

\documentclass{acm_proc_article-sp}

\begin{document}

\title{Comparison between centralized and distributed systems and how they cope with different threat models\titlenote{Title is work in progress.}}
%\subtitle{[Extended Abstract]
%\titlenote{A full version of this paper is available as
%\textit{Author's Guide to Preparing ACM SIG Proceedings Using
%\LaTeX$2_\epsilon$\ and BibTeX} at
%\texttt{www.acm.org/eaddress.htm}}}
%
% You need the command \numberofauthors to handle the 'placement
% and alignment' of the authors beneath the title.
%
% For aesthetic reasons, we recommend 'three authors at a time'
% i.e. three 'name/affiliation blocks' be placed beneath the title.
%
% NOTE: You are NOT restricted in how many 'rows' of
% "name/affiliations" may appear. We just ask that you restrict
% the number of 'columns' to three.
%
% Because of the available 'opening page real-estate'
% we ask you to refrain from putting more than six authors
% (two rows with three columns) beneath the article title.
% More than six makes the first-page appear very cluttered indeed.
%
% Use the \alignauthor commands to handle the names
% and affiliations for an 'aesthetic maximum' of six authors.
% Add names, affiliations, addresses for
% the seventh etc. author(s) as the argument for the
% \additionalauthors command.
% These 'additional authors' will be output/set for you
% without further effort on your part as the last section in
% the body of your article BEFORE References or any Appendices.

\numberofauthors{2} 
% I've updated the number of authers ~ Christoffer 2012-10-14

%  in this sample file, there are a *total*
% of EIGHT authors. SIX appear on the 'first-page' (for formatting
% reasons) and the remaining two appear in the \additionalauthors section.
%
\author{
% You can go ahead and credit any number of authors here,
% e.g. one 'row of three' or two rows (consisting of one row of three
% and a second row of one, two or three).
%
% The command \alignauthor (no curly braces needed) should
% precede each author name, affiliation/snail-mail address and
% e-mail address. Additionally, tag each line of
% affiliation/address with \affaddr, and tag the
% e-mail address with \email.
%
% 1st. author
\alignauthor
Predrag Filipovikj\\
       \email{predragku@gmail.com}
% 2nd. author
\alignauthor
Christoffer Holmstedt\\
       \email{christoffer.holmstedt@gmail.com}
}
% There's nothing stopping you putting the seventh, eighth, etc.
% author on the opening page (as the 'third row') but we ask,
% for aesthetic reasons that you place these 'additional authors'
% in the \additional authors block, viz.
% \additionalauthors{Additional authors: John Smith (The Th{\o}rv{\"a}ld Group,
%email: {\texttt{jsmith@affiliation.org}}) and Julius P.~Kumquat
%(The Kumquat Consortium, email: {\texttt{jpkumquat@consortium.net}}).}
%\date{30 July 1999}
% Just remember to make sure that the TOTAL number of authors
% is the number that will appear on the first page PLUS the
% number that will appear in the \additionalauthors section.

\maketitle
\begin{abstract}
Through time computer networks have evolved into fault tolerant systems, that have substantial level of reliability. Still they are far from perfect, and we can find numerous potential threats that can affect their proper functioning. 

Constructing a high quality threat models and studying their effect on different types of networks can be a very complex process that requires a substantial amount of time and financial founds. Instead of that, in this paper we tend to desing rather simple threat models with high probbability of occurance, and by applying them to a specific network architecture or applications we try to analyze their effect on centralized and decentralized computer networks.

The main focus of this paper are centralized and decentralized network architectures. In order to test their resistance to threats, we have introduced three threat models, which are designed in that fashion to capture specific aspects of the particular architecture. We put our emphasis on how centralized and decentralized computer network deal with infrastructure and data integrity threats. By applying our threat models, we can see the potential flows into the architecture, which will be used to summarize which network architecture is more tolerant to that specific threat. 
As a conclusion we will summarize how our threat models changed the characteristics of the centralized and decentralized networks and propose best solution in situations with high probability of particular threat model.
\end{abstract}

% A category with the (minimum) three required fields
\category{H.4}{CHANGE THIS Information Systems Applications}{Miscellaneous}
%A category including the fourth, optional field follows...
\category{D.2.8}{CHANGE THIS Software Engineering}{Metrics}[complexity measures, performance measures]

\terms{Theory}

\keywords{ACM proceedings, \LaTeX, text tagging} % NOT required for Proceedings

\newpage
\section{Introduction}
Example citation \cite{example}.


\newpage
\section{Method}
How did you find the information so you could say what the state of the art is?
Example citation \cite{example}.


\section{Threat models}
This report will focus on different threat models and how a centralised architecture and a distributed architecture (peer-to-peer) will deal with each threat.
In this chapter all threat models will be listed with references to events that has occured in the past that makes each threat real.

\subsection{A natural disaster}
This is the first threat model which will focus on infrastructure failure.

A natural disaster occurs that wipes out the infrastructure that powers the cell phone towers as well as other communication systems that under normal circumstances are available such as DSL, fiber or landlines.
\begin{itemize}
  \item The only devices left operating are the ones with battery such as laptops and cell phones.
  \item These devices have one or more features that allows them to connect to other similiar devices, example of this could be WiFi, Bluetooth, IR or USB drive.
\end{itemize}

\subsection{Private information is compromised}
This is the second threat model which will focus on privacy concerns in social networks.

Social media such as Facebook, Twitter and Google Plus becomes bigger and bigger for each day that passes by.
It's not easy to stay away from social media today as the services have become a good part of everyday life.
If you're not in, you are an outsider. 
With the increasing amount of private data about all of the services' respective users what would happened if that information was compromised.

\begin{itemize}
  \item Users share private information with friends and families and believes that this information will be restricted to only the people they allow read access to.
  \item The private information is someday accessed by someone that shouldn't have access to it. 
\end{itemize}

\subsection{Censorship}
The last threat model that we are going to put focus on is the censorship of the data on the internet.

As the number of user grows every day, the amount of data present into the internet infrastructure grows with even bigger rate. Users tend to share different type of information, without taking much care about the contents of it. There are some types of information like pornography, piracy, hate-speech  that should be monitored and even banned. But sometimes this becomes excuse for the authorities to sniff someones data and to decide what is right and what is wrong.

\begin{itemize}
  \item Users want to share some information outside their physical reach.
  \item Users data has been compromised by some third parities (authorities) that do not want that information do be spread.
  \item Banned global services in purpose of propaganda.
\end{itemize}


\section{Natural disaster}
During the last decade the evolution of cell phones has gone from just being mobile phones to todays "smartphones".
The smart phones can do a wide variety tasks depending on which applications are installed.
The last years rise in smart phone usage has made the use of social media expand to all parts in society \cite{taylor2012}.

The penetration rate of social media has almost made the use of social media by governments and different agencies a must, for communication to its citizen, especially in crises.
As an example The Queensland Police Service in Australia used their facebook page successfully during several floodings and thunderstorms in 2011.
Their success is shown by the number of "likes" of their facebook page that grow "[...] from under 20 000 to over 160 000 within three days of flash flooding [...]" \cite{taylor2012}.
A like can mean a lot of things, everything from a person just want to appreciate what they (the recipient of the like) do to someone who really found the information posted valuable.

Other ways of using social media is by citizens that can create their own networks and communicate with each other.
During the floodings in Queensland and Victoria (Australia) in 2010 and 2011 several of these networks were created \cite{bird2012}.
The reports that came in to these networks and in some cases republished (shared by more than the original author) were often about the latest news from agencies and traditional news such as TV and radio.
Though this was not the only way of using the different networks in social media, one main usage of social media is the communication between citizen e.g. citizen that has moved from one area to another can report about road blocks as well as which road are available to travel on.
The updates between citizens often include more local information than the regular news reports \cite{bird2012}.

In \textit{"The role of social media as psychological first aid as a support to community resilience building"} \cite{taylor2012} the conclusion is drawn that social media can function as "psychological first aid (PFA)" \cite{taylor2012}. 
\begin{quote}
\textit{"The role of social media in
this context is not to replace face-to-face support or contact, or to replace official warning services, but it can expand capacity to deliver information, extend the reach of official messages and limit the psychological damage caused by rumours and sensationalised media reporting."} \cite{taylor2012}
\end{quote}
This shows that social media such as facebook has gone from being just a community where you exchange basic status updates and comments to be have a role in emergencies.

Furthermore Lawson et al \cite{lawson2012} writes about cell phones as a part of disaster preparedness.
The qualitative research conducted took place in Haiti where several natural disasters occur every year.
During disasters landlines most often goes down which could happened to cellular towers as well but is not as usual according to one participant in the survey.
A more likely scenario is that the cell networks get overcrowded after a disaster when everyone with a cell phone tries to contact relatives and friends.
The overcrowded traffic makes it near impossible to get through with regular phone calls but data traffic does.

With this background it's clear that in the world today technology such as cell phones and services such as Facebook, Twitter and Google plus have a big role when it comes to getting a community back on track after a natural disaster.
It's now time to recap the threat model mentioned in earlier chapter.
The following is the scenario given:
\begin{itemize}
  \item The only devices left operating are the ones with battery such as laptops and cell phones.
  \item These devices have one or more features that allows them to connect to other similar devices, example of this could be WiFi, Bluetooth, IR or USB drive.
\end{itemize}
The question now arise how would a centralised system coup with this as well as how would a decentralised system do?

Centralised systems such as cell phone networks as well centralised services such as Facebook depend on backup systems.
Cell towers requires emergency power from a reserve power unit or a generator of some kind.
The cell towers also needs a connection out to the cell network for communication to be possible.
In a centralised system like cell networks it's clear that these towers are a weak point in the chain, exactly how weak is hard to say.

On standby network operators have "cellular on wheels" or COWs in short.
COWs are basically a mobile cellular tower.
These COWs are moved in preparation of a thunderstorm to the locations that will most likely be struck hardest.
COWs help the network to get up and running as soon as possible, even if it's not at its full capacity it is at least some traffic that goes through \cite{swartz1999}.

The next step in the chain is the services which are most likely hosted somewhere far away from the disaster area and still up and running.
Even if the disaster would strike where the services are hosted their will probably be redundancy geographically in another part of the world which would still be operational.

Research has in the last years been conducted with the goal to use peer-to-peer network for services such as sharing content, video and music \cite{web:tribler}, as well as creating a social network \cite{web:peerson}.
It's a logical step to move towards a peer-to-peer social media network which could be used in the case of emergency.
If a peer-to-peer or ad-hoc network were possible to set up straight after a natural disaster between the survivors the initial communication could commence straight away.

Another possibility with a peer-to-peer network is that it could be designed in a delay tolerant fashion.
This would mean that information could spread to others over the course of hours or days even if the sender and receiver doesn't have access to the network at the same point in time.
As an example to demonstrate this imagine a disaster strikes a village.
They use an already existing peer-to-peer network or set up a new one to organise the recovery work in the village.
After a few days the first rescue workers reach the village and to get an up to date picture of the situation they connect to the local peer-to-peer network and get all information needed in a few seconds. 
The network in itself would in this case act as a logging system of all the work that has been done and what needs to be done in a decentralised fashion.


% Section about Natural disaster threat model \cite{taylor2012,lawson2012,bird2012}


\section{Privacy in Social Network Services}
As stated in earlier chapters social media is on the rise.
Most of these networks are centralised e.g. Facebook, Twitter and Google plus.
Users share any kind of private information in the belief that it is their own information only accessible by friends that they know.
This is though far from true in many cases.
As an example social media websites search through all information submitted by users to be able to target the user with tailored ads.
This is not appreciated by most americans as a telephone survey conducted in 2009 conclude \cite{turow2009}.
It's not far fetched in this scenario to believe that similar results would be acquired in other parts of the world.

Other studies show that "institutional" privacy e.g. the scenario above when a company analysis its users behaviour and shared information is not the main concern \cite{raynes-goldie2010}.
Instead the users in the study thought that "social" privacy is more important.
Social privacy include scenarios from friend requests (from the "wrong" person e.g. the users boss or teacher) to posting daily messages that should be suitable for all friends.
Some participants in the study was troubled by the lack of control of shared data on facebook.
As an example it's not often that the same message is perceived in the same way by your boss and your best friend.
To take the example one step further a message posted during a night out with friends in the weekend could be misinterpreted by the users boss and have consequences the next workday.

It's important to remember that what we perceive as problems with privacy in different services one day may not be a problem the next day.
In 2006 Facebook introduced the news feed function or the Facebook wall as it's called today.
With this release all information available in every users network of friends got aggregated to one simple list of actions taken during the last period of time.
The information was available all the time but not that easy to get an overview of, now the news feed became that overview and it became the main feature of Facebook.
Today, in 2012, facebook groups form for all kinds of reason and this was also the case back in 2006.
In a few days the group "Students Against Facebook News Feed" had over 700 000 people \cite{boyd2008}.
This clearly shows that privacy is a main concern for many people though privacy can come in different shapes and forms.
The news feed is still the main feature on Facebook today after some privacy settings were added.

One interesting feature about privacy is that users that know about loopholes in security will use them even if the same action against themselves is not appreciated.
Raynes-Goldie found that participants in her study used direct URLs to images on Facebook that was normally hidden behind privacy settings \cite{raynes-goldie2010}.

With the centralised systems covered it's time to look at the decentralised systems.
The basics of a decentralised system are the lack of central nodes, a central authority that controls the network and the ability to scale very well.
In \textit{"Friendly Surveillance - A New Adversary Model for Privacy in Decentralized Online Social Networks"} \cite{greschbach2012} the authors defines services such as Facebook and Google plus to be a "Social Network Service (SNS)" and decentralised versions are named "Decentralised Online Social Network (DSON)".
It is important to remember that traditional SNS have been online for several years but it's only recent years that researcher has started to look at DSONs.
The move from a SNS to a DSON effectively remove the earlier mentioned institutional privacy problems but on the downside also introduce several new problems.

The first problem is the question on how to store data.
With the lack of a central node to store the information it must be stored somewhere else.
A first step would be to encrypt all data and store it on all your friends computers so at least when you go offline others can access it through your friends that are online.
The problem here now arise that a friend of you could monitor the traffic that comes in and request particular pieces of information.
Even if the information is encrypted metadata is available such as how much disk space the information takes and how often a specific node requests similar data.
With the information about disk space usage your friend could find out if it's text, music or videos that are shared.
With the information about how frequently a node requests some information and during which hours of the day the sender could compose a pretty clear picture on the behaviour of that node \cite{greschbach2012}.
It's also in this area Greschbach et al \cite{greschbach2012} comes to the conclusion that a good technical solution that doesn't give away too much information in metadata is needed.

The second question is about trust.
How do a user trust another user.
In a centralised system users trust the centralised service as an authority or "third-party" so if user A sees user B online user A can in some sense be sure that it's user B that is online.
This is because users indirectly approve of the authentication system used by the service.
In decentralised systems there is no authority that vouch for someone else instead decentralised system can use some kind of reputation.
User B could choose to improve the reputation of User A.
If then User C comes in contact with User A the reputation set by User B (together with many others) gives a representation on how well User A behaves in the network and how much trust can be given to User A.
The problem with a system like this is that reputation or trust is often depending on context \cite{mondal2006}.
User B might trust User A when it comes to talking about some soccer games that took place yesterday but User B might not trust User A when it comes to talking about private information such as medical status.
Another aspect is that the context also differs between users.
A generalization in a reputation system as explained above clearly has it drawbacks and it's in this area , "effective trust models" Mondal et al \cite{mondal2006} suggests that the research community can contribute.

Tightly coupled with trust is also accountability.
If one user in the network has bad behaviour it will probably be easy to deny that user further access to whatever resource is requested.
Though one of the main features of many decentralised systems is the possibility for a user to be completely anonymous.
The possibility to be completely anonymous makes it possible for the user that has bad behaviour in the network to act as a new user when the first one has been blocked by a noticeable amount of users.
In the end to introduce accountability to a decentralised system might not what you want to do due to low amount of malicious users \cite{mondal2006}.



\section{Censorship}
The rate of the growth of Internet along with the amount of information present within its infrastructure, becomes a potential threat for the certain groups of people, governments and other national institutions. Users tend to share any kind of data on the Internet, and some of them can be harmful to the other users. The primary concerns related to the data present on the Internet are the hate-speech, pornography, piracy, etc. The bad information should be prevented, but this process must not lead to restricting the users from their right for freedom of speech. As it usually goes with potential flaws, prevention of the harmful data is being used as an excuse for limiting the freedom of speech. 

Vast number of recent examples of Internet censorship used to limit the freedom of speech can be found in China \cite{canaves}, Libya \cite{dianotti2011} and many other countries. 
This leads to defining a potential threat model that could affect the computer networks and limit the information flow between the users - a Censorship Threat Model.

In this paper, authors want to present how the Censorship Threat Model can affect the computer networks, and how centralized and decentralized architectures deal with the it. The easiest way to present that is through study of the most popular WWW social networking sites, like Facebook, Twitter and Google Plus as examples of centralized systems and try to introduce an efficient way for dealing with censorship in this types of network architecture. On the other side there are many applications on the Internet that relay on distributed architecture. Those are known as P2P (peer to peer) applications, and by studying them we can see how decentralized architecture deal with the Censorship Threat Model.

During the demonstrations in Egypt against their government and the president \cite{web:scialnetworkcriticalmass}, people were using the most popular networks for sharing information. In circumstances like this, it is vital to have a good communication, in order to organize the demonstrations and to gain on more masses. 
Because people were using a centralized networking sites, like Facebook, Twitter, YouTube, and etc., it was very easy for the regime to track down their data and censure them. This is because in centralized systems like those mentioned, we have a central point called server. 

In order to establish communication between each other, peers or the clients must first refer to the central point, in order to become aware of each other. The server in this case represents the central point, where every single bit of information must pass through it. In a system like this, the process of censorship is pretty straight forward - just inspect all the traffic that is coming in to the specific central point and going out from it, and filter the data that you do not want to be present in the network. 

One possible solution for not being censored by inspection of your data is to use different algorithms for encrypting your data. Yes, this is a way of protecting the integrity of the data, but there is no guarantee that this encrypted data will not be censored. If one client in this model is suspicious, it is easy to filter his encrypted data. Because all the data must go to the central point, and the infrastructure in which the data travel is usually in government property or they have a mechanism to gain access to it \cite{dianotti2011}, they can discard every suspicious packet on its way to the server.

If there is too much of undesirable traffic into the system, it is not practical to filter all of the data coming from numerous suspicious sources, rather than that one can block the access to the central point. If the central point has being cut off the system, the clients can no longer communicate since they are not aware of each others presence, and the network is practically disabled - there is no data flow into the system. One particular example of this is that back in March 2011 Google has accused Chinese authorities for blocking the Google mail service and look like the it was a problem with Gmail service. The story was released to public by one of the most prominent American daily newspapers, The New York Times. In the text there were official statements from Google. "There is no issue on our side; we have checked extensively", they said. "This is a government blockage, carefully designed to look like the problem is with Gmail." \cite{web:newyorktimes}

Another weak point in this type of network architecture is centralized storage of the user data. All of the user data must be stored on the central point of the system in order to be shared or give access to it to the other users of the same system. This feature of the centralized systems is also affected by the Censorship Threat Model, and there is still no efficient way for this kind of architecture to deal with this kind of threat.
In contrast to the centralized network model we have the decentralized or distributed network architecture. As previously stated the most significant difference of this network architecture compared with the centralized one is the absence of a central point for supporting the communication with the clients. 

In order to present how decentralized network architectures can potentially avoid the Censorship Threat Model, we are going to discuss Freenet \cite{clarke2001} a typical pure peer-to-peer \cite{web:peertopeer} type of decentralized network architecture, which relies on anonymity. When it was reviled for the first time, its founders argued that the anonymity of the system provides a true freedom of speech. The core building units of this network architecture are equipotent peers. In order to be able to communicate with each other, the peers or in this case they are called "nodes" need to maintain local table with addresses to their neighbors. All of the nodes in the system have equal privileges, and therefore there is no need for a single central point. The biggest advantage of this kind of architectures in avoiding Censorship Threat Model is the absence of a central point. This means that in the system, there is no longer a single point of failure or censorship.  

As described in paper \cite{clarke2001}, the architecture of this application impose storing the data between the users. In the initial version of the application, the data was distributed across the system, but it still was stored in one piece. Even though the data is encrypted, because it is stored in one place there was a potential risk that this data could be decrypted. If the data were to be some sensitive information, its holder may be subjected to prosecution. For that purpose, a multiple algorithms have been proposed in order to improve the robustness and resistance to censorship. One of the most popular one was proposed in Regine Endsuleit and Thilo Mie paper \cite{endsuleit2006}. 

In their paper they proposed a new distributed way of storing the data across the system. Instead of the traditional way of storing the data in one place, now the data has been divided into small chunks, that are then distributed and stored among the clients. In addition to decentralized storing of the information, also there is a lot of redundancy in the system. A single chink of information is stored on multiple clients. This means that a single chunk of information is available from many places. All these features are introduced into the network in order to increase the fault tolerance of the system to various threat models, among which is our Censorship Threat Model. This implies that in order to retrieve some information, one must first find all the chunks and then try to reassemble them. Duo to the decentralization of the system, and distributed information among the peers, this network architecture with these characteristics is very hard to be censored.


\section{Conclusions}
In this report no new information is presented.
The report is an overview on how centralised and decentralised systems cope with different threats.
The threats are composed in three different threat models defining three different scenarios.
It's clear that defining just one threat model and making sure that all corner cases are covered is a demanding task and a time consuming process in itself.
When writing this report it was early on decided that the focus should be put on the comparing centralised and decentralised system instead of trying to define the perfect threat models.
For future research defining the different threat models in more detail is a first step.

The comparisons made show that centralised and decentralised systems have their pros and cons.
No one can say that centralised or decentralised systems will be better nor worse in dealing with all types of threats.
In the threat models defined in this report decentralised systems have a small advantage.
As an example to this in the aftermath of a natural disaster a community can on their own set up a decentralised network without having to wait for someone to fix a centralised node e.g. setting up a temporary cellular tower.
This would mean that a more organised and controlled recovery would start sooner with a decentralised network than a centralised network.

As a last note it is important to remember what is trying to be achieved.
In the threat model about privacy it is a must to define what type of privacy you discuss because it is a big difference between "institutional privacy" and "social privacy".

%\end{document}  % This is where a 'short' article might terminate

% Just comment this out if we don't need it.
\input{acknowledgment.tex}

%
% The following two commands are all you need in the
% initial runs of your .tex file to
% produce the bibliography for the citations in your paper.
\bibliographystyle{abbrv}
\bibliography{sigproc}  % sigproc.bib is the name of the Bibliography in this case
% You must have a proper ".bib" file
%  and remember to run:
% latex bibtex latex latex
% to resolve all references
%
% ACM needs 'a single self-contained file'!
%
%APPENDICES are optional
%\balancecolumns
%\appendix
%Appendix A
%\section{Headings in Appendices}
%The rules about hierarchical headings discussed above for
%the body of the article are different in the appendices.
%In the \textbf{appendix} environment, the command
%\textbf{section} is used to
%indicate the start of each Appendix, with alphabetic order
%designation (i.e. the first is A, the second B, etc.) and
%%a title (if you include one).  So, if you need
%hierarchical structure
%\textit{within} an Appendix, start with \textbf{subsection} as the
%highest level. Here is an outline of the body of this
%document in Appendix-appropriate form:
%\subsection{Introduction}
%\subsection{The Body of the Paper}
%\subsubsection{Type Changes and  Special Characters}
%\subsubsection{Math Equations}
%\paragraph{Inline (In-text) Equations}
%\paragraph{Display Equations}
%\subsubsection{Citations}
%\subsubsection{Tables}
%\subsubsection{Figures}
%\subsubsection{Theorem-like Constructs}
%\subsubsection*{A Caveat for the \TeX\ Expert}
%\subsection{Conclusions}
%\subsection{Acknowledgments}
%\subsection{Additional Authors}
%This section is inserted by \LaTeX; you do not insert it.
%You just add the names and information in the
%\texttt{{\char'134}additionalauthors} command at the start
%of the document.
%\subsection{References}
%Generated by bibtex from your ~.bib file.  Run latex,
%then bibtex, then latex twice (to resolve references)
%to create the ~.bbl file.  Insert that ~.bbl file into
%the .tex source file and comment out
%the command \texttt{{\char'134}thebibliography}.
% This next section command marks the start of
% Appendix B, and does not continue the present hierarchy
%\section{More Help for the Hardy}
%The acm\_proc\_article-sp document class file itself is chock-full of succinct
%and helpful comments.  If you consider yourself a moderately
%experienced to expert user of \LaTeX, you may find reading
%it useful but please remember not to change it.
\balancecolumns
% That's all folks!
\end{document}
